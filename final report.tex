\documentclass[12pt]{article}

\usepackage[margin=1 in]{geometry}

%Use Times New Roman font
%\usepackage{pslatex}

% allow text colors
\usepackage[usenames,dvipsnames,svgnames,table,x11names]{xcolor}

%allow double-spacing
\usepackage{setspace}

%enable support for figures
\usepackage{graphicx}

\usepackage{rotating}
\usepackage{multirow}

\usepackage{url}

%handle filenames better
\usepackage{grffile}

%math
%\usepackage{mathtools}
\usepackage{amsmath}
\usepackage{amsfonts}
\usepackage{lipsum}

%control figure movement
\usepackage{placeins}

%center captions
\usepackage{caption}

%circuits and units
\usepackage[free-standing-units]{siunitx}
\DeclareSIUnit\vrms{\volt{}_{RMS}}
%\usepackage[americanvoltages,americancurrents]{circuitikz}
%\usetikzlibrary{calc}

% make scientific notation easy
\providecommand{\e}[1]{\ensuremath{\times 10^{#1}}}

% include pdf documents
\usepackage{pdfpages}

%for loops
\usepackage{pgffor}

%indent after new sections
\usepackage{indentfirst}

\usepackage{verbatim}

% allow greek letters outside math mode. \text<name of letter>
%\usepackage{textgreek}

% margin editing
\usepackage{changepage}

% braket notation, other goodies
\usepackage{physics}
\usepackage{braket}

%source code
\usepackage{listings}

%hyperlink support
\usepackage{hyperref}

% bibtex support
\usepackage{natbib}

%settings for Python code
%\lstset{basicstyle=\footnotesize\ttfamily,
%commentstyle=\color{OliveGreen},
%keywordstyle=\color{blue},
%tabsize=4,
%numbers=left,
%stringstyle=\color{red},
%language=Python,
%inputencoding=utf8,
%extendedchars=true,
%showstringspaces=flase,
% }

\begin{document}
\singlespacing
\title{Final Report\\
Experimental Physics 380N}
\date{Jul 31 2015}
\author{Robert Rosati}
\maketitle

\begin{abstract}
\par A driver circuit for piezoelectric positioners for scanning probe microscopes (SPMs) was constructed for Professor de Lozanne's lab at UT-Austin, following \cite{Chen12}. The circuit can drive a Pan-type piezoelectric stick-slip motor with up to six teeth, driven individually in sequence while slipping, and in union while sticking \cite{Pan99}. This driver circuit is novel in its simplicity and low cost -- while traditionally a sawtooth waveform has been used to drive stick-slip motors, any smoothly increasing and sharply decreasing waveform will work. A simple exponential rise/decay (i.e., capacitively charging/discharging a piezoelectric element) is much easier to generate electronically, and avoids the expensive high-voltage op-amps needed to create a sawtooth. A sequence of sharp falls can be generated cost-effectively by cascaded relays. Together, these advances give a positioner driver circuit greater than an order of magnitude less expensive (\$40) than traditional approaches. I successfully constructed the circuit and used it to walk an atomic force microscope (AFM).
\end{abstract}

\doublespacing
\section{Design}
\par Stick-slip motors operate by forcing a stationary piezoelectric element to expand and contract in contact with a moving element. If the piezo moves more quickly in one direction than the other, and the coefficients of kinetic and static friction are sufficiently different, the moving element can be driven in a particular direction. The piezo alternately sticks in the direction of travel and slips in the opposite direction. Typical movement during one step of such a motor is on the order of tens of nanometers.
\par Pan-type stick-slip motors operate under a similar principle, except the slipping motion is broken into several stages. The piezo element is divided into several independently diven teeth -- these are made to slip one-by-one, until all teeth have slipped. The teeth are then driven together while sticking (see figure \ref{fig:panmotor}). This motor design is widely considered more robust than the simple stick-slip described above, because as each tooth attempts to slip it contends with the static frictional force of all other teeth.
\begin{figure}[ht]
\centering
%\includegraphics[scale=0.8]{pan-type-motor}
\caption{A Pan-type stick-slip motor with four teeth \cite{Pan99}.}
\label{fig:panmotor}
\end{figure}

\par Any waveform smooth enough to force sticking and sharp enough to force slipping will work -- traditionally this is a sawtooth, generated by several operational amplifiers \cite{Horowitz89}. A simpler alternative is to generate the sharp charge/discharge of the piezo by switching a relay, and the smooth sections of the waveform by capacatively charging/discharging the piezo through a resistor. Electronically, piezoelectric materials can be considered capacitors -- a piezo expanded under high voltage will force current to flow through a resistor during its exponential discharge.
\par The circuit to generate this waveform consists of three main stages: an oscillator, voltage converters, and a relay cascade (see schematic).
\begin{figure}[ht]
\centering
%\includegraphics[scale=0.4]{dpdt-relay}
\caption{A double-pole, double-throw (DPDT) relay with a pinout identical to the ones used in my implementation. When a current is passed from pin 1 to pin 8, pin 3 becomes connected to pin 4, and pin 6 to pin 5.}
\label{fig:relay}
\end{figure}
The voltage converters provide a $5\volt $ rail for the relays and a high voltage rail ($72\volt $ to $144\volt $) for the piezo elements. The oscillator, when high, triggers the relay cascade. DPDT relays (see figure \ref{fig:relay}) are used so that one side of the relay can propagate the cascade signal, while the other can switch the piezo voltages. Each relay takes roughly $5 \milli\second$ to fully switch, which gives the piezos ample time to fully respond (typical response time of a few hundred microseconds) \cite{Horowitz89}.

\par By switching whether the piezos slowly charge and rapidly discharge or vice-versa, the direction of travel can be controlled.`

\section{Prototyping}
\par In order to test and refine the above design, the circuit was first constructed on a breadboard. Once I ordered or found the necessary parts (see bill of materials), I began tuning the output waveform. As it began to resemble the ideal shape, the circuit was live-tested to drive the positioner of an atomic force microscope.
\par Some modifications to the design described in the previous section were ultimately made during the prototyping of the circuit.
Undesirable characteristics of the output waveform were eliminated by varying the value of the piezo discharge resistor (see schematic), further filtering the voltage multiplier's output (increasing $\textrm{C5}$), and varying the oscillator timing.
Eventually, I decided to add a frequency selector switch ($5\hertz$ and $0.5\hertz$, for final approach) and two selectable piezo voltages, at half and full voltage quadrupler output. A waveform inverter switch (to choose the direction of the stick-slip motors) was also implemented.
\par Instead of purchasing a more expensive transistor with windings for both $36 \vrms$ and a voltage appropriate for rectification to a $5 \volt$, I chose to purchase the less expensive and more readily available transformer with only a $36 \vrms$ secondary, and used a spare USB phone charger to provide the necessary relay voltage.
\section{Final Construction}
\par After the prototyping and modifications described above, I began constructing a more permanent version of the circuit.
I laid out the components roughly how they were arranged on the breadboard, taking care to isolate the high voltages traces at least several millimeters away from lower voltage traces. This isolation should prevent arcing, even with a higher voltage transformer and humid air. The practical piezo voltage limit is set by the voltage quadrupler's filter capacitor, which is rated for $200\volt$. Replacing it would allow voltages up to $400\volt$ (set by the $100\volt$ rating of the quadrupler capacitors).
The components were soldered to a protoboard, a PCB with some traces already printed in short segments. I attempted to use the existing traces as much as possible, however some solder bridging between adjacent traces and jumper wires were still necessary.
\begin{figure}[ht]
\centering
%\includegraphics[scale=0.1]{{2015-07-28 01.05.49}.jpg}\\
%\includegraphics[scale=0.1]{{2015-07-28 01.06.01}.jpg}
\caption{The front and back of the fully populated (save for the transformer and the $5\volt$ source) protoboard. Note that the piezo outputs are routed in \textcolor{orange}{\textbf{ORANGE}} to the pin connector in the lower-right. $\textrm{S0}$ through $\textrm{S3}$ (see schematic) are out of frame.}
\label{fig:boardpic}
\end{figure}
\begin{figure}[ht]
\centering
%\includegraphics[scale=0.1]{{2015-07-28 01.05.12}.jpg}\\
%\includegraphics[scale=0.1]{{2015-07-28 02.22.15}.jpg}
\caption{(above) The fully populated protoboard mounted in its enclosure, note the small black box (the $5\volt$ source) to the side. (below) The fully enclosed circuit, in its aluminum housing, with operating instructions temporarily taped on.}
\label{fig:encpic}
\end{figure}
\par I chose to mount the protoboard on three bolts, through holes drilled through both the board and the enclosure.
The traces nearest the bolt holes were left unused and plastic washers were placed between the mounting nuts to isolate the entire board electrically from its case.
The main transformer was also secured with two of the bolts.
Switches were installed with jumper wires at the necessary locations (see figure \ref{fig:boardpic})-- my intention was to make each switch's purpose immediate and obvious.
One of my main goals was to ensure the board was removable from the case without a soldering iron -- I chose switches that could be mounted with only a nut and washers for this reason.
The output voltages were wired to a 7-pin (6 outputs plus ground) connector, which cannot be removed from the case once soldered to the outputs. It is connected to the board through a pin connector, however, to facilitate easy board removal. I included a full schematic inside the enclosure as well, to make maintenance or upgrades as painless as possible.
\par Unfortunately the AFM was not in working order when I completed this circuit, so I was unable to test the fully constructed version directly. However, I tested all voltage outputs and they gave identical results to the breadboard circuit which had successfully walked the AFM. I therefore feel confident that the stick-slip driver circuit is in working order.

\section{Schematic}
\par For completeness, a full schematic of the circuit as it was constructed is reproduced below. All capacitors are $1\micro\farad$, and the piezo discharge resistors are $1\mega\ohm$. Piezo capacitance of the AFM used for testing was typically $\sim 100\nano\farad$. Externally visible switches are numbered $\textrm{S0}$ through $\textrm{S3}$, from left to right. Outputs are across the piezoelectric elements, $\textrm{P1}$-$\textrm{P6}$.
%\begin{figure}
%\includegraphics[scale=0.9,angle=270,origin=c]{piezo-driver-2}
%\caption{Full schematic of the constructed circuit. All capacitors are $1\micro\farad$, and the piezo discharge resistors are $1\mega\ohm$. Typical piezo capacitance is ~$100\nano\farad$. Externally visible switches are numbered $S0$ through $S3$, from left to right. Outputs are across the piezoelectric elements, $P1$-$P6$.}
%\end{figure}
%\includepdf[pages={1},angle=90]{piezo-driver-2}

\section{Bill of Materials}
\par Unfortunately, complete pricing on many of my components is unknown, as they were prepurchased by the department's electronics shop. What follows is an accurate BOM to my knowledge.
\begin{table}[ht!]
\centering
\begin{tabular}{lll}
\textbf{Schematic identifier}&\textbf{Description}& \textbf{Price}   \\
C1-C4                & ceramic, rated to 100V       & ?       \\
C5                   & rated to 200V                & ?       \\
D1-D4                & rated to 600V                & ?       \\
Q1                   & npn transistor, small signal & ?       \\
R1,R2                & 1\mega\ohm, carbon film      & ?       \\
R3;R4                & 91\kilo\ohm, carbon film     & ?       \\
R5                   & 2.2\kilo\ohm, carbon film    & ?       \\
R6-R11               & 1 \mega\ohm, carbon film     & ?       \\
RLY1-RLY6            & Axicom V23079A1001B301       & 3.45 ea \\
Relay sockets        & 5-pin, 0.05" center spacing  & 0.77 ea \\
S0,S3                & SPDT switch                  & ?       \\
S1,S2                & DPDT switch                  & ?       \\
T1                   & 120V:36V, center tapped      & 11.96   \\
U1                   & 555 IC                       & ?       \\
U2                   & USB phone charger, Motorola  & ?       \\
enclosure            & Aluminum, 7"x5"x3"           & 13.23      
\end{tabular}
\caption{Bill of materials for the constructed circuit. Parts with unknown value were obtained from the physics electronics shop or were already in Prof de Lozanne's electronics repository.}
\label{tab:BOM}
\end{table}
\clearpage
\bibliographystyle{plain}
\bibliography{research}

\end{document}
